\usepackage[top=1in, bottom=1in, left=1in, right=1in]{geometry}


\usepackage{listings}
\usepackage{color}
\usepackage[english]{babel}

\definecolor{mylgrey}{rgb}{0.9,0.9,0.9}
\definecolor{mydgrey}{rgb}{0.5,0.5,0.5}
\definecolor{mygreen}{rgb}{0,0.6,0}
\definecolor{mygray}{rgb}{0.5,0.5,0.5}
\definecolor{mymauve}{rgb}{0.58,0,0.82}
 
% \lstset{
%         basicstyle=\ttfamily\small, 
%         keywordstyle=\color{keywords},
%         commentstyle=\color{comments},
%         stringstyle=\color{red},
%         showstringspaces=false,
%         identifierstyle=\color{green},
%         procnamekeys={def,class}}


\lstset{ %
  language=Python, 
  basicstyle=\ttfamily\small, 
  backgroundcolor=\color{mylgrey},   % choose the background color
  xleftmargin=5mm, 
  xrightmargin=10mm, 
  frame=l, 
  rulesep=5mm,
  framerule=1pt,
  showstringspaces=false,
  rulecolor=\color{mydgrey},
  breaklines=true,                 % automatic line breaking only at whitespace
  numbers=left,  
  numberstyle=\small\color{mydgrey},
  captionpos=b,                    % sets the caption-position to bottom
  commentstyle=\color{mygreen},    % comment style
  escapeinside={\%*}{*)},          % if you want to add LaTeX within your code
  keywordstyle=\color{blue},       % keyword style
  stringstyle=\color{mymauve},     % string literal style
}

\newcommand{\mylisting}[2][]{%
\lstinputlisting[caption={\texttt{\detokenize{#2}}},#1]{#2}
}
% \mylisting[linerange={1-2,4-5}]{example1.py}

\newcommand{\makesubincl}{
\noindent\fcolorbox{gray}{white}{
\parbox[t]{0.95\linewidth}{
\centering
\parbox[t]{0.9\linewidth}{
    \vspace*{1em}
\begin{description}
\item[Collaborators] \varAnsColl
\item[Sources] \varAnsSrc
\item[Late days] $( \varAnsLtP + \varAnsLtC ) / 3$
\end{description}
    \vspace*{1em}
    }
}}}

\date{\today}