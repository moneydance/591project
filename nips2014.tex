\documentclass{article} % For LaTeX2e
\usepackage{nips14submit_e,times}
\usepackage{hyperref}
\usepackage{url}
%\documentstyle[nips14submit_09,times,art10]{article} % For LaTeX 2.09
\usepackage[top=1in, bottom=1in, left=1in, right=1in]{geometry}


\usepackage{listings}
\usepackage{color}
\usepackage[english]{babel}

\definecolor{mylgrey}{rgb}{0.9,0.9,0.9}
\definecolor{mydgrey}{rgb}{0.5,0.5,0.5}
\definecolor{mygreen}{rgb}{0,0.6,0}
\definecolor{mygray}{rgb}{0.5,0.5,0.5}
\definecolor{mymauve}{rgb}{0.58,0,0.82}
 
% \lstset{
%         basicstyle=\ttfamily\small, 
%         keywordstyle=\color{keywords},
%         commentstyle=\color{comments},
%         stringstyle=\color{red},
%         showstringspaces=false,
%         identifierstyle=\color{green},
%         procnamekeys={def,class}}


\lstset{ %
  language=Python, 
  basicstyle=\ttfamily\small, 
  backgroundcolor=\color{mylgrey},   % choose the background color
  xleftmargin=5mm, 
  xrightmargin=10mm, 
  frame=l, 
  rulesep=5mm,
  framerule=1pt,
  showstringspaces=false,
  rulecolor=\color{mydgrey},
  breaklines=true,                 % automatic line breaking only at whitespace
  numbers=left,  
  numberstyle=\small\color{mydgrey},
  captionpos=b,                    % sets the caption-position to bottom
  commentstyle=\color{mygreen},    % comment style
  escapeinside={\%*}{*)},          % if you want to add LaTeX within your code
  keywordstyle=\color{blue},       % keyword style
  stringstyle=\color{mymauve},     % string literal style
}

\newcommand{\mylisting}[2][]{%
\lstinputlisting[caption={\texttt{\detokenize{#2}}},#1]{#2}
}
% \mylisting[linerange={1-2,4-5}]{example1.py}

\newcommand{\makesubincl}{
\noindent\fcolorbox{gray}{white}{
\parbox[t]{0.95\linewidth}{
\centering
\parbox[t]{0.9\linewidth}{
    \vspace*{1em}
\begin{description}
\item[Collaborators] \varAnsColl
\item[Sources] \varAnsSrc
\item[Late days] $( \varAnsLtP + \varAnsLtC ) / 3$
\end{description}
    \vspace*{1em}
    }
}}}

\date{\today}


\title{Detecting Infected Hosts and Domains}

%% Not sure why this is not working!?
\author{
Tim  Lim\\
\and
Dan Brown \\
\and
Ben Pusey \\
\AND
https://github.com/moneydance/591project
}

% The \author macro works with any number of authors. There are two commands
% used to separate the names and addresses of multiple authors: \And and \AND.
%
% Using \And between authors leaves it to \LaTeX{} to determine where to break
% the lines. Using \AND forces a linebreak at that point. So, if \LaTeX{}
% puts 3 of 4 authors names on the first line, and the last on the second
% line, try using \AND instead of \And before the third author name.

\newcommand{\fix}{\marginpar{FIX}}
\newcommand{\new}{\marginpar{NEW}}

\nipsfinalcopy % Uncomment for camera-ready version

\begin{document}


\maketitle

\begin{abstract}
Advanced persistent threats have become a major concern for IT professional around the world. Their stealthy distributed nature makes them difficult 
to identify and remove. However because the connections between backdoors and command and control centers are bipartite, certain graph theory
techniques can be used to identify malicious domains, and infected hosts
\end{abstract}

\section{Introduction}
Cyber security is an ever-changing field. As malware becomes more advance so do methods of detection. Recently a sophisticated attack called and Advanced Persistent Threat (APT) has emerged. These 



%% \subsection{Keywords for paper submission}
%% Your NIPS paper can be submitted with any of the following keywords (more than one keyword is possible for each paper):

%% \begin{verbatim}
%% Clustering
%% Control and Reinforcement Learning
%% Feature Selection
%% Learning Theory
%% Margins and Boosting
%% Other Algorithms and Architectures
%% Other Applications

%% \end{verbatim}


\section{Methodology}
\label{method}

The entirety of the data is stored in the form of DNS logs. In terms of a broad overview, our methodology is simply the following: parse the logs, filter the parse result, build a graph, analyze that graph, get a set of domains,
look into the interactions involving those domains, and then report those possibly infected domains and hosts. \\
For parsing, we check for all responses that contain the letter "A" separated by a space from all other characters or words. The appearance of the letter A, in this form, is a reference to an address record, meaning 
there has been some kind of interaction between a domain as well as a host. So, we save that record, and transform it into part of the overall final graph, in which each node represents (labelled as an IP-address) either a domain or a host,
and the edge between them represents an interaction between that host and domain (the edge contains the actual string of the interaction). Since a host can have multiple interactions with the same domain, the final graph
is actually a multi-way-graph, which can contain multiple edges between a domain and host. \\
Once our final graph has been constructed from all the responses found in the log, we use Degree Centrality to find out the amount of hosts that
are connected to (or have contacted) each of the domains. We then look at which of these domains have a low-degree centrality (small amount of connections), then use those domains to aid in the execution of a Belief Propagation algorithm. \\
Once the Belief Propagation finishes, it will return a sub-network of possibly infected hosts and domains. We can then look into those possibly infected nodes, and check for suspicious activity. 

\subsection{Rare Domains}
Many infected domains tend to be ones that are specifically targeted by malicious hosts. These domains are specifically chosen by malicious hosts because they tend to be ones that have a small amount connections. \\
A small amount of connections means a small amount of requests are processed by the domain, which potentially means less security, less tracking of hosts and their interactions with the domain, and an overall 
less chance of the malicious hosts being discovered and dealt with for committing malicious actions. Because of this type of low-detection chance environment, domains with low connections could serve as a front 
for all kinds of malicious behaviour (they could be C\&C, also known as Command and Control domains) and thus, are worth looking into. \\
Determining which domains have a small amount of connections is quite simple- this is where degree centrality becomes useful. We can simply run a degree centrality algorithm on the entire graph and return all 
the domains that have a low degree centrality (small number of connections). However, a small degree centrality can also simply mean that a domain is not popular, in which case malicious behaviour cannot be inferred. 
Instead, since we are looking for possible C\&C domains, we create a bound and look for all domains that have a degree centrality within that bound. \\
Now, since this bound can potentially return a large portion of the graph, we then use some of the domains returned by this centrality as seeds for the BP algorithm (Belief Propagation), while the rest will be later 
combined with the results of the BP algorithm, then filtered and returned as the final sub-network of possibly infected hosts and domains. 

\subsection{Belief Propagation}
We use the Belief Propagation algorithm to return a greater and somewhat more accurate set of possible C\&C nodes.
The premise of the BP algorithm is simple: imagine we have a set of nodes. All those nodes have a label, and each node's neighbor has an idea of the possibility of its neighboring nodes having a certain label or message. Based on this 
possibility, a similarity score is assigned to each node. In the case of networks and malicious behavior, we can think of that message as Malware, and when this algorithm is done running, it returns a score that shows the influence one 
node might have over other nodes, which potentially represents malicious hosts that have an influence on infected domains as well as other infected hosts (possible C\&C pairs). \\
The main advantage of this algorithm is that we allow the network itself to find some of the possible C\&C pairs, which is much more efficient and more accurate than using other means to achieve the same result. And that result will be in
the form of nodes which have a high score according to the BP algorithm. These nodes are then combined with the nodes from the previous centrality algorithm (duplicates removed), and are then filtered by checking for malicious activity.

We include a code listing for our belief propagation algorithm, written in Python:
\begin{lstlisting}
def belief_propagation(G, hosts, doms, threshold=0.5, max_iter=100):
	"""
	find new suspicious hosts/domains given sets of seed domains and hosts.
	"""
	assert type(hosts) == type(doms) == set
	if not hosts:
		lo = 5
		hi = 15           # arbitrary
		for node in G:
			if len(doms) > 20: break
			if re.search(r'[a-z]', node) is None: continue
			if G.degree(node) > lo and G.degree(node) < hi and detect(G, node):
				doms |= {node}
	
		for dom in doms: hosts |= get_hosts(G,dom)
	raredoms = set()
	for host in hosts: raredoms |= rare_domains(G, host)
	
	count = 0
	while len(raredoms) > 0:
		count += 1
		if count > max_iter:
			print 'max_iter reached! breaking...'
			break
		
		newdoms = set()
		remove = set()
		for dom in raredoms:
			if dom in doms:
				continue
			if detect(G, dom):
				newdoms |= {dom}
				remove |= {dom}
		raredoms -= remove
		
		if not newdoms:
			score = {}  # dict, not set
			for dom in raredoms:
				if dom in doms:
					continue
				score[dom] = compute_score(G, dom, hosts)
			# get domain of max score
			if score == {}: break
			max_dom = reduce(lambda x,y: x if score[x] >= score[y] else y, score)
			if score[max_dom] >= threshold:
				newdoms |= {dom}
			else:
				print 'no domain with score above threshold! breaking.. \n'
				break
		else:
			doms |= newdoms
			for dom in newdoms: hosts |= get_hosts(G,dom)
			for host in hosts: raredoms |= rare_domains(G,host)
	if len(raredoms) == 0 : print "broke because no more rare domains! \n"
	return hosts, doms
\end{lstlisting}

\subsection{Checking for suspicious activity}
Within the edges of the suspicious domains are their interactions with hosts (some of which may also be suspicious). These interactions are in the form of the strings from the log file. Determining whether these strings have any irregular or
malicious behaviour is not a simple task as the data can look deceptively normal (as though a non-infected host and/ or domain were interacting). There is, for this reason, no definitive way to be certain of malicious behaviour; however, 
certain checks can have a decent chance of detecting an actual infected host. Our method employs one of these checks. \\
The CNAME operator refers to the renaming of an IP-address. Multiple instances of this CNAME within the response of a domain mean that multiple mappings exist for the same IP-address. Normally, a renaming of an IP-address does not happen often,
and does not occur more than twice within the same response. So, seeing CNAME appear more often than the norm is indicative of irregular and potentially malicious behaviour (this could, for instance, be multiple redirects caused by Malware). \\
For this reason, we simply parse and check each response string to see if it contains a certain number of CNAME operators.
%% add code of CNAME for presentation here
Doing this check enhances the accuracy of our returned C\&C domains, and increases the likelihood that some of those hosts and domains are actually infected. 

\section{Experiments}
\label{exp}
Due to the sheer size of the data itself, running, let alone reading, all of it would require access to computers or large amounts of time that we simply did not possess. For this reason, it is a more suitable approach to 
look at a smaller amount of time, for example - a day's worth of data, instead of all the data that comprises several months of network information. Even just one day of data could contain a large amount of infected domains 
and hosts, and would pose quite a challenge in order to find those that may or may not be infected.

\subsection{A day's worth of data}
Many malicious hubs, malicious hosts, infected domains, and other infected hosts can appear within the time-frame of a day. Domains that were completely fine one day, can become infected on the next. So it is not unreasonable to test
and do experiments involving just one day's worth of DNS log data. And in fact, looking at one day might be more beneficial than looking at an entire month, especially if later experiments also focus on just a day of data but later on 
in that month. \\
This could potentially form a basis for creating a more in-depth, focused, and localized area of interest within the possible infected sub-network(s) of the entire network, which could be more useful than a broad analysis
involving an entire month of data.

\section{Results}
\label{res}
Here are the results of the experiments:
...

\begin{verbatim}
logs
\end{verbatim}

\subsection{Interpretation}
We run our code with the following settings, and output the sets of domains and hosts we find, as well as edge information between them, to a log file:
\begin{lstlisting}
infile = '2013-03-17'
num_edges = 500000
edgelist = parseToGraph.parse(infile, num_edges=num_edges)
G = construct_graph(edgelist)
hosts, doms = belief_propagation(G, set(), set(), threshold=0.7)
\end{lstlisting}

\begin{verbatim}
74.92.39.47
74.92.32.18
74.92.74.110
74.92.47.73
74.92.67.20
74.92.174.204
74.92.74.157
74.92.38.152
74.92.210.24
74.92.169.178
74.92.14.160
74.92.111.62
74.92.62.205
74.92.241.121
74.92.36.107
74.92.56.80
74.92.39.83
74.92.69.169
74.92.10.60
74.92.169.56
74.92.42.46
74.92.155.178
184.202.111.41
74.92.96.151
74.92.148.15
184.202.20.220
74.92.163.47
74.92.171.191
74.92.39.79
74.92.30.112
74.92.220.19
74.92.30.116
74.92.140.160
74.92.136.59
74.92.39.53
92.160.212.105
74.92.23.86
74.92.185.4
74.92.4.27
74.92.107.121
74.92.245.101
252.90.80.26
74.92.180.150
74.92.8.144
74.92.243.44
74.92.36.115
74.92.74.11
74.92.46.106
74.92.118.27
74.92.94.183
74.92.123.9
74.92.138.187
74.92.111.222
74.92.176.102
58.229.128.1
74.92.77.104
74.92.169.10
74.92.14.27
74.92.208.220
74.92.231.233
74.92.172.7
74.92.100.219
74.92.226.78
74.92.169.110
74.92.4.74
74.92.215.80
74.92.80.56
74.92.179.46
74.92.147.238
74.92.114.49
74.92.208.178
74.92.50.119
74.92.175.32
58.208.125.7
58.208.125.6
74.92.224.26
74.92.12.8
58.229.45.32
74.92.190.162
74.92.125.38
74.92.240.231
74.92.49.12
74.92.54.53
74.92.26.44
74.92.80.130
74.92.100.40
74.92.74.94
74.92.4.214
74.92.185.33
74.92.50.52
74.92.182.142
184.202.159.108
74.92.43.108
74.92.157.35
74.92.38.3
74.92.139.55
74.92.94.190
184.202.84.131
74.92.77.153
74.92.50.31
74.92.77.115
74.92.132.75
184.202.138.101
74.92.248.83
74.92.83.155
74.92.250.98
74.92.240.78
74.92.255.26
74.92.151.144
74.92.12.52
74.92.89.91
74.92.65.85
74.92.81.93
74.92.25.140
74.92.81.90
74.92.64.45
184.202.152.7
74.92.38.238
74.92.79.62
74.92.195.204
74.92.150.213
74.92.125.134
74.92.156.31
184.202.58.166
74.92.213.113

ump.thumb.dimly.wad
fulfil.johannes.wad
hastening.nullify.wad
suites.dusted.wad
cot.ledger.wad
rattiest.add.wad
lam.preponderances.wad
peddler.vet.wad
requisition.vanishing.wad
fa.fop.plot.hated.wad
gluey.jeans.tad.wad
shrimp.ab-z7g6r.noe
fa.ad-.plot.hated.wad
u.refurnished.wad
cot.fireproof.wad
oberon.aacire9v9zf.wad
pestilence.jocasta.noe
ob.enhanced.wad
step.hated.wad
braved.racier.wad
rev.guileless.wad
pit.clashed.rimbaud.wad
blantyre.superstitions.wad
bacteriologists.sapped.console.val
i.refurnished.wad
pours.comb.co.rd
dick.ably.ox.wad
ump.frill.dimly.wad
pert.la.agt.slyly.wad
sn.hated.wad
cot.preponderances.wad
na.blantyre.superstitions.wad
cot.ki.wad
ming.foam.inching.hated.wad
blantyre.rev.console.noe
occlusion.rimbaud.wad
rev.faint.wad
ripple.nails.wad
stoppering.conversationalists.wad
aaaa8y5807h1ayfufc0u7.tamra.b.tridents.noe
cot.jeans.tad.wad
rho.sedation.relentless.wad
kempis.jeans.tad.wad
pm.ohio.wad
cot.sledge.wad
fa.rue.plot.hated.wad
gent.ti.ty.friend.noe
did.toothy.co.rd
shadowiest.tad.wad
\end{verbatim}

An example printout of the edge data encoded between a host and domain in this list is:
\begin{verbatim}
For nodes 74.92.32.18, braved.racier.wad:
date : 2013-03-17 00:05:38.806021
data: 
? braved.racier.wad A
! braved.racier.wad CNAME collective.racier.wad
! collective.racier.wad CNAME collective.racier.wad.eco.racier.wad.friend.noe
! collective.racier.wad.eco.racier.wad.friend.noe CNAME marathon.racier.wad.wryness.noe
! marathon.racier.wad.wryness.noe CNAME moody.g.eyeballing.noe
! moody.g.eyeballing.noe A 59.195.249.218

date : 2013-03-17 00:10:38.720933
data: 
? braved.racier.wad A
! braved.racier.wad CNAME collective.racier.wad
! collective.racier.wad CNAME collective.racier.wad.eco.racier.wad.friend.noe
! collective.racier.wad.eco.racier.wad.friend.noe CNAME marathon.racier.wad.wryness.noe
! marathon.racier.wad.wryness.noe CNAME moody.g.eyeballing.noe
! moody.g.eyeballing.noe A 59.195.249.218

date : 2013-03-17 00:15:38.834693
data: 
? braved.racier.wad A
! braved.racier.wad CNAME collective.racier.wad
! collective.racier.wad CNAME collective.racier.wad.eco.racier.wad.friend.noe
! collective.racier.wad.eco.racier.wad.friend.noe CNAME marathon.racier.wad.wryness.noe
! marathon.racier.wad.wryness.noe CNAME moody.g.eyeballing.noe
! moody.g.eyeballing.noe A 59.195.249.218

date : 2013-03-17 00:20:38.755377
data: 
? braved.racier.wad A
! braved.racier.wad CNAME collective.racier.wad
! collective.racier.wad CNAME collective.racier.wad.eco.racier.wad.friend.noe
! collective.racier.wad.eco.racier.wad.friend.noe CNAME marathon.racier.wad.wryness.noe
! marathon.racier.wad.wryness.noe CNAME moody.g.eyeballing.noe
! moody.g.eyeballing.noe A 59.195.249.218

date : 2013-03-17 00:25:38.757728
data: 
? braved.racier.wad A
! braved.racier.wad CNAME collective.racier.wad
! collective.racier.wad CNAME collective.racier.wad.eco.racier.wad.friend.noe
! collective.racier.wad.eco.racier.wad.friend.noe CNAME marathon.racier.wad.wryness.noe
! marathon.racier.wad.wryness.noe CNAME moody.g.eyeballing.noe
! moody.g.eyeballing.noe A 59.195.249.218

date : 2013-03-17 00:30:39.008724
data: 
? braved.racier.wad A
! braved.racier.wad CNAME collective.racier.wad
! collective.racier.wad CNAME collective.racier.wad.eco.racier.wad.friend.noe
! collective.racier.wad.eco.racier.wad.friend.noe CNAME marathon.racier.wad.wryness.noe
! marathon.racier.wad.wryness.noe CNAME moody.g.eyeballing.noe
! moody.g.eyeballing.noe A 59.195.249.218
\end{verbatim}
\section{Conclusion}
\label{conclusion}
In conclusion...



\subsubsection*{Acknowledgments}

Use unnumbered third level headings for the acknowledgments. All
acknowledgments go at the end of the paper. Do not include 
acknowledgments in the anonymized submission, only in the 
final paper. 
...

\subsubsection*{References}
%%Examples *change them for the love of...*: 
Examples: \\

\small{
[1] Alexander, J.A. \& Mozer, M.C. (1995) Template-based algorithms
for connectionist rule extraction. In G. Tesauro, D. S. Touretzky
and T.K. Leen (eds.), {\it Advances in Neural Information Processing
Systems 7}, pp. 609-616. Cambridge, MA: MIT Press.

[2] Bower, J.M. \& Beeman, D. (1995) {\it The Book of GENESIS: Exploring
Realistic Neural Models with the GEneral NEural SImulation System.}
New York: TELOS/Springer-Verlag.

[3] Hasselmo, M.E., Schnell, E. \& Barkai, E. (1995) Dynamics of learning
and recall at excitatory recurrent synapses and cholinergic modulation
in rat hippocampal region CA3. {\it Journal of Neuroscience}
{\bf 15}(7):5249-5262.
}

\end{document}
